\documentclass[12pt, a4paper, oneside]{ctexbook}
\usepackage[a4paper, left=3cm, right=3cm, top=2.5cm, bottom=2.5cm]{geometry} % 调整边距
\usepackage{amsmath, amsthm, amssymb, bm, graphicx, hyperref, mathrsfs, fancyhdr, pifont, float}


\title{{\Huge{\textbf{技术经济与工程管理}}}\\——2024秋季学期}
\author{ZZY1234321}
\date{\today}
\linespread{1.5}
\newtheorem{theorem}{定理}[section]
\newtheorem{definition}[theorem]{定义}
\newtheorem{lemma}[theorem]{引理}
\newtheorem{corollary}[theorem]{推论}
\newtheorem{example}[theorem]{例}
\newtheorem{proposition}[theorem]{命题}

\begin{document}

\pagestyle{fancy}
\fancyhf{}
\fancyhead[L]{\leftmark}
\fancyhead[R]{\thepage}
\renewcommand{\headheight}{15pt} % 使用有效的格式
\setlength{\headsep}{20pt} % 设置页眉与正文之间的距离

\maketitle

\pagenumbering{roman}
\setcounter{page}{1}

\begin{center}
    \Huge\textbf{前言}
\end{center}~\

本笔记参考课程老师的PPT,以及b站up主“滹沱河樵夫”发布的“工程经济与管理”合集。如有侵权,可联系本人。笔记解释权归作者本人所有。此笔记正在Github持续更新中,方便的话麻烦点个star,感谢支持。

~\\
\begin{flushright}
    \begin{tabular}{c}
        ZZY1234321\\
        \today
    \end{tabular}
\end{flushright}

\newpage
\pagenumbering{Roman}
\setcounter{page}{1}
\tableofcontents
\newpage
\setcounter{page}{1}
\pagenumbering{arabic}

\chapter{现金流量的概念}

\section{基本概念}

\subsection{投资}
投资的广义概念是指有目的的经济行为,泛指企业的一切资金分配与运用行为,是企业为了获取所期望的报酬而投入某项计划的资源,所投入的资源包括资金,也包括人力、技术或信息等其它资源。

投资的狭义概念是指为实现某建设项目而预先垫付的资金。对于一般的工业投资项目来说,总投资包括建设投资和生产经营所需要的资金、建设期的借款利息。
本课程常用的投资是投资的狭义概念。

包括:\textbf{固定资产投资、流动资金投资、无形资产投资、递延资产投资和预备费用、(建设期利息)}。

固定资产投资活动按其工作内容和实现方式分为:“建筑安装工程”;“设备、工具、器具购置”;“其他费用”三个部分。

\subsection{费用}
支出中凡是与本企业的经营有关的各项耗费称为费用。

\subsection{现金流出}
对一个系统而言,凡在某一时点上流出系统的资金或货币量,如投资、费用等。

\subsection{现金流入}
对一个系统而言,凡在某一时点上流入系统的资金或货币量,如销售收入等。
$$
\mbox{净现金流量(}NCF\mbox{)}=\mbox{现金流入(}CI\mbox{)}-\mbox{现金流出(}CO\mbox{)}
$$

\subsection{现金流量}
各个时点上实际的资金流出或资金流入(现金流入、现金流出及净现金流量的统称)。

\section{现金流量图的说明}
横轴是时间轴,每个间隔表示一个时间单位,点称为时点,标注时间序号的时点通常是该时间序号所表示的年份的年末。

与横轴相连的垂直线,箭头向上表示现金流入,向下表示现金流出,长短与现金流量绝对值的大小成比例,箭头处一般应标明金额。

一般情况,时间单位为年,假设投资发生在年初,销售收入、经营成本及残值回收等均发生在年末。
$$
\mbox{固定资产净值 = 固定资产原值 - 累计折旧}
$$

期末残值:项目寿命期结束时固定资产的残余价值。

\section{各种资产分类}
\subsubsection{流动资产投资}
储备资金(原材料、燃料等)、生产资金(在制品、半成品、待摊费用)、成品资金(产成品、外购品等)、结算资金(应收、预付帐款等)、货币资金(备用金、现金、银行存款等)

\subsection{无形资产}
包括专利、著作权、版权、商标、专有技术等。其价值在服务期内逐年摊销,摊销费计入成本。

\subsection{递延资产}
递延资产,是指本身没有交换价值,不可转让,一经发生就已消耗,但能为企业创造未来收益,并能从未来收益的会计期间抵补的各项支出。包括开办费、租赁固定资产改良费、固定资产装璜、装修费等中在规定年限内平均摊销,摊销费计入成本。

\section{折旧的计算}
\subsection{直线法}
\subsubsection{平均年限法}
$$\mbox{年折旧率}=(1-\mbox{预计净残值率)/折旧年限}$$
$$\mbox{年折旧额=固定资产原值} \times \mbox{年折旧率}$$

\subsubsection{工作量法}
$$\mbox{单位工作量折旧额=(原值-残值)/额定总工作量}$$
$$\mbox{年折旧额=单位工作量折旧额} \times \mbox{年工作量}$$

\subsubsection{双倍余额递减法}
$$\mbox{年折旧率=}2/\mbox{折旧年限} \times 100\%$$
$$\mbox{年折旧额=固定资产净值} \times \mbox{年折旧率}$$

\chapter{资金的时间价值与等值计算}

\section{资金的时间价值的概念}

\subsection{概念}

不同时间发生的等额资金在价值上的差别,就称为资金的时间价值。

\subsection{资金具有时间价值的内涵:}

资金在生产与交换过程中由于有劳动者的劳动使之产生了增值。

资金的时间价值是对放弃现时消费的必要补偿。

\subsection{影响资金使用的因素}

\begin{itemize}
    \item 投资收益率
    \item 风险
    \item 通货膨胀
\end{itemize}

\subsection{资金等值的概念}

资金等值是指在不同时点绝对值不等而价值相等的资金。

在一个或几个项目中,投资或收益往往发生在不同的时间,于是就必须按照一定的利率将这些投资或收益折算到某一个相同的时点,这一过程就是等值计算。

\section{利息和利率}
\subsection{利息$I_n$}
贷款人向借款人让渡资金使用权而得到的一种报酬,也是借款人占用资金所付的代价。

某人存入银行一笔资金,存期\textbf{5年}(\textbf{计息期}),\textbf{每年}(\textbf{计息周期(每次计算利息的时间单位)})计息一次,整个存储期间共计息\textbf{5次}(\textbf{n:计息周期次数})。

\subsection{利息的表示方式}
\subsubsection{(1)绝对数表示:}

$$I_n = F_n - P$$

$F_n$:经过$n$个计息周期后的本利和。
\subsubsection{(2)相对数表示:}
指单位本金经过一个计息周期的利息数额。

$$i = \frac{I_1}{p} \times 100\%$$

$P$:利率。

\subsection{利率$i$}

一定时期内(一年、半年、月、季度,即一个计息周期)利息总额与本金(借贷金额)的比率。

$$\mbox{利率} = \frac{\mbox{期利息}(I_1)}{\mbox{本金}(P)} \times 100\%$$

利率是单位本金经过一个计息周期后的增殖额。(年利率、半年利率、月利率,……)

\subsection{利息计算方法}
\subsubsection{(1)单利法}

仅对本金计息,利息不再生利息。
$$I_n= p \cdot n\cdot i$$
$$F_n=P \cdot (1+i \cdot n)$$

\subsubsection{(2)复利法}

核心:以本金与累计利息之和为计息基数,即“利上加利”、“利滚利”、“驴打滚”。

技术经济分析中时间价值一般采用复利法,充分反映资金的时间价值。

$$F_n=P \cdot (1+i)^n$$

\begin{itemize}
    \item $P$:原始本金;
    \item $i$:一个计息周期利率;
    \item $n$:计息期内计息周期次数。
\end{itemize}

\subsubsection{(3)差异分析}
单利计息:对资金时间价值的考虑不完整,利息没转入记息基数;

复利计息:充分反映资金的时间价值。


\subsection{名义利率和实际利率}
\subsubsection{(1)名义利率}
在技术经济分析中,复利计算通常以年为计息周期。但在实际经济活动中,计息周期有年、半年、季、月等多种,这样出现了不同计息周期的利率换算问题。

如:按月计算利息,且其月利率为1\%,通常称为“年利率12\%,每月计息一次”。这个年利率12\%称为“\textbf{名义利率}”。

名义利率的计算:
$$\mbox{名义利率}(r) = \mbox{每一计息周期的利率} \times \mbox{一年中的计息周期的次数}(m)$$
$$r/m = \mbox{每一计息周期利率}$$

\subsubsection{(2)实际利率}
\textbf{实际利率:}当一年内多次复利计息时,按照一年内获得的利息与年初本金之比计算出的年利率为实际年利率。

实际利率的计算:
$$\mbox{实际利率}(i)=\mbox{一年内按复利计息的利息总额}/\mbox{年初本金}$$\\
举例:
本金1000元,每月复利计息一次,月利率1\%,一年后的本利和为:
$$F = 1000 \times (1+1\%)^{12}= 1126.8 \mbox{(元)}$$
实际利率为:
$$\frac{1126.8-1000}{1000} \times 100\%=12.68\%$$

\subsubsection{(3)名义利率和实际利率公式推导}
复利计息,一年后本利和为:
$$F=P(1+r/m)^m$$

一年内的利息额为:
$$I=F-P=P \times (1+r/m)^m-P=P \times [(1+r/m)^m-1]$$

实际利率为:
$$i=(F-P)/P=\frac{P \times (1+r/m)^m-p}{p}=(1+r/m)^m-1$$

若计息周期为一年,$m = 1$,则$i = r$,即实际利率=名义利率。

若连续计息, $m \to \infty$,则:
$$i = \lim_{m \to \infty}(1+\frac{r}{m})^m-1=e^r-1$$

若名义年利率为12\%,以下各种情况下,实际年利率等于多少?

\textcircled{1}按年计息,m=1
$$i = (1+ r/m)^m-1=(1+0.12/1)^1-1=12\%$$

\textcircled{2}按半年计息,m=2
$$i = (1+ r/m)^m-1=(1+0.12/2)^2-1=12.36\%$$

\textcircled{3}按季度计息,m=4
$$i = (1+ r/m)^m-1=(1+0.12/4)^4-1=12.55\%$$

\textcircled{4}按月计息,m=12
$$i = (1+ r/m)^m-1=(1+0.12/12)^{12}-1=12.68\%$$

\textcircled{5}按连续计息,m=$\infty$
$$i=e^r-1=e^{0.12}-1=12.75\%$$
\textbf{例1:}某人将1000元存入银行,定期3年,年利率12\%,3年期满,按复利计算,期满后他可以从银行得到[填空1]元。\\
\textbf{答案:}\\
实际利率:$i=(1+\frac{12\%}{12})^{12}-1=1.01^{12}-1=12.68\%$\\
应归还本利和:$20 \times (1+12.68\%)^3=20 \times 1.4308=28.62$(万元)\\
\textbf{例2:}下列说法正确的是\\
A. 名义年利率条件下,一年的利息额是按单利计算
\\
B. 实际年利率条件下,一年的利息额是按单利计算
\\
C. 实际年利率条件下,一年的利息额是按复利计算
\\
D. 名义年利率条件下,一年的利息额是按复利计算\\
\textbf{答案:}AC\\
\textbf{例3:}(判断题)若一年中复利计息的次数大于1,则年实际利率小于年名义利率。\\
\textbf{答案:}错误\\
差异原因分析:
$$\mbox{年利率}i=\frac{\mbox{一年的利息额}}{\mbox{年初本金}} \times 100\%$$
名义年利率:一年的利息额是按单利计算的;\\
实际年利率:一年的利息额是按复利计算的;\\
当一年中多次计息时(即$m>1$时),两者产生差异。


\section{资金的等值计算}

\subsection{基本概念}

资金的时间价值:在不同的时间付出或得到同样数额的资金在价值上是不等的,即资金的价值会随着时间发生变化。

评价技术的经济效果,不仅要考虑现金流入、流出的数额,还要考虑每笔现金流量发生的时间。

不同时间发生的等额资金在价值上的差别称为资金的时间价值。也即资金在生产和流通过程中随着时间推移而产生的增值。对资金使用者来讲,是使用资金的成本。

某项目投资100万元,建成投产后,每年可得利润20万元,即为100万元在特定生产经营活动中所产生的时间价值。

\subsubsection{决定资金等值的因素}
\begin{itemize}
    \item 资金数额
    \item 资金发生的时刻
    \item 利率:关键因素
\end{itemize}

资金随着时间的推移,其价值会增加。从消费者角度讲,是对放弃现期消费的补偿。

\subsubsection{资金的时间价值引发的问题}

\begin{figure}[H]
    \centering
    \includegraphics[width=0.8\textwidth]{image/资金时间价值引发的问题.png}
    \caption{资金时间价值引发的问题}
    \label{fig:3}
\end{figure}

不能直接比较不同时间点的资金的价值大小;

需要利用等值换算,将不同时点的资金换算到同一时点进行比较。

\subsubsection{资金时间价值表现形式}
\begin{itemize}
    \item 资金投入生产或流通领域产生的增值称为利润(Profit)或收益(Income)。
    \item 把资金存入银行或向银行借贷所得到或付出的增值额称为利息(Interest)。
\end{itemize}

\subsubsection{资金时间价值衡量尺度}
(1)绝对尺度:体现资金时间价值绝对量的多少。利息、利润或收益;

(2)相对尺度:反映资金时间价值相对量的大小。利息率、利润率或收益率。

\subsection{资金等值}
\subsubsection{概念}
在考虑资金时间价值的情况下,在不同的时间绝对值数额不等的若干资金,如果具有相同的价值,则为等值的资金。

例如,年初的100元和年底的110元,在单利10\%的情况下是等值的。

\subsubsection{资金等值概念的意义}
利用等值概念,通过等值计算,可以知道某一时点上的资金金额在其他时点上的价值。可以把不同时点发生的资金金额换算到同一时点进行价值比较。
\subsubsection{相关概念}
等值资金:在利率一定的条件下,我们把不同时间(时期、时点)上绝对数额不等,而经济价值相等的若干资金,称为等值资金。

资金等值计算:利用资金等值原理,我们可以把某一时间点上的资金值,按照所给定的利率换算为与之等值的另一时间点的资金值,这一换算过程称为资金的等值计算。

折现(贴现Discount):把将来某一时间点上的资金值换算成现在时间点上的等值资金值。

折现率($i$,Discount Rate):进行资金等值计算中使用的反映资金时间价值的参数叫折现率。(如利率、收益率)

现值($P$,Present Value):“现值”并非专指一笔资金“现在”的价值,它是一个相对的概念。一般地说,将$t+k$个时点上发生的资金折现到第$t$个时点,所得的等值金额就是$t+k$个时点上的资金金额的现值。

终值($F$,Future Value):与现值等价的将来某时点的资金值称为“终值”。

等年值($A$,年金,Annual Value):分期等额收支的资金值。\\
\textbf{例题:}(判断)折现是指将来时点的资金金额换算成期初时点的资金金额这一过程。\\
\textbf{答案:}错误。

\section{资金等值计算公式(重点掌握)}
按照现金流的不同,等值公式可以分为:

一次收付类型:两个

等额分付类型:四个(另引申四个)

等差序列类型(自学,了解)

等比序列类型(自学,了解)

\subsection{一次收付型}
是指所分析系统的现金流量,无论是流入还是流出,均在一个时间点上一次发生。包括:

一次收付终值(未来值)公式

一次收付现值公式

\subsubsection{一次收付终值公式($P \to F$)}

\begin{figure}[H]
    \centering
    \includegraphics[width=0.8\textwidth]{image/一次收付终值现金流量图.png}
    \caption{一次收付终值现金流量图}
    \label{fig:3}
\end{figure}

已知$P$,求$F$,即求$P$在$n$年后的等值资金$F$

复利终值计算公式:\\
(与复利计息本利和计算公式相同)
$$F=P \cdot (1+i)^n$$

或

$$F=P \cdot (F/P,i,n)$$



\textbf{整付终值计算公式}

已知期初投资为$P$,利率为$i$,求第n年末收回本利$F$。
$$F=P(1+i)^n$$

$(1+i)^n$称为整付终值系数,记为$(F/P,i,n)$。

$(F/P,i,n)$中,$F$为欲求因素,其余为已知因素。复利终值系数可查表,如图\ref{fig:5}:

\begin{figure}[H]
    \centering
    \includegraphics[width=\textwidth]{image/复利终值系数表.png}
    \caption{复利终值系数表}
    \label{fig:5}
\end{figure}

\subsubsection{一次收付现值公式($F \to P$)}
即在已知利率$i$的条件下,要想在$n$期期末得到资金$F$,期初应一次投入多少资金?其现金流量图如图\ref{fig:6}所示。

\begin{figure}[H]
    \centering
    \includegraphics[width=0.8\textwidth]{image/一次收付现值现金流量图.png}
    \caption{一次收付现值现金流量图}
    \label{fig:6}
\end{figure}

\textbf{复利现值计算公式}
$$P=\frac{F}{(1+i)^n}=F \cdot (1+i)^{-n}$$(一次支付终值公式的逆运算)\\
或
$$P=F \cdot (P/F,i,n)$$(复利现值系数与复利终值系数互为倒数)\\
\textbf{例1:}给你一次选择机会,你可以现在获得1000元,也可以在5年以后的今天获得1300元,在年利率为8\%的情况下,你将做何种选择?\\
\textbf{答案:}$P=F/(1+0.08)^5=1300/(1+0.08)^5=884.76$\\
由于在年利率为8\%时, 5年后的1300元等值于现在的884.76元,小于同一时点的1000元,故应该选择现在获得1000元。\\
\textbf{例2:}某人把1000元存入银行,设年利率为6\%,复利计息,5年后全部提出,共可得多少元?

$F=P(1+i)^n=1000 \times (F/P,6\%,5)=1000 \times 1.338=1338$(元)\\
\textbf{例3:}某环保制造企业计划建造一条生产线,预计5年后需要资金1000万元,设年利率为10\%,复利计息,问现需要存入银行多少资金?

$P=F(1+i)^{-n}=1000 \times (P/F,10\%,5)=1000 \times 0.6209=620.9$(万元)

\subsection{等额分付系列公式}

含义:
等额分付是多次支付形式的一种,多次支付指现金流入和流出在多个时点上发生,而不是集中在某个时点上。当现金流序列是连续的,且数额相等,则称为等额系列现金流。
\begin{itemize}
    \item 等额分付年金终值公式
    \item 等额分付偿债基金公式
    \item 等额分付年金现值公式
    \item 等额分付资本回收公式
\end{itemize}

\subsubsection{等额收付年金终值公式($A \to F$)}

若每期期末支付同等数额的资金$A$,在利率为$i$的情况下,$n$期后的未来值应该是多少?其现金流量图如图\ref{fig:7}所示。

\begin{figure}[H]
    \centering
    \includegraphics[width=0.8\textwidth]{image/等额收付终值现金流量图.jpg}
    \caption{等额收付终值现金流量图}
    \label{fig:7}
\end{figure}

\textbf{年金终值公式:}
$$F=A(1+i)^{n-1}+A(1+i)^{n-2}+\mbox{…}+A(1+i)+A=A \sum_{i=1}^{n} (1+i)^{n-1}=A[\frac{(1+i)^n-1}{i}]$$(利用等比级数求和公式)\\
或
$$F = A(F / A, i, n)$$(年金终值系数)\\
\textbf{例题:}某人从现在开始的三年内每年年末存入银行1000元,存款利率为10\%,复利计息,计算第三年年末该人银行账户的余额。\\
\textbf{答案:}$F=A \cdot (F/A,10\%,3)=1000 \times 3.310 = 3310$(元)\\
\textbf{引申:}如果上例改为“三年内每年年初存入银行1000元”,其他条件不变,则第三年年末该人银行账户的余额为多少?\\
\textbf{答案:}$F_d=A(F/A,10\%,3) \times (1+i)=1000 \times 3.310 \times (1+10\%)=3641$(元)\\

\textbf{先付年金}

$F_d= A \cdot ( F / A , i , n ) \cdot (1 + i)$

\begin{figure}[H]
    \centering
    \includegraphics[width=0.8\textwidth]{image/先付年金图示.jpg}
    \caption{先付年金图示}
    \label{fig:8}
\end{figure}

\subsubsection{等额分付偿债基金公式($F \to A$)}
等额收付偿债基金公式是等额分付终值公式的逆运算。即已知终值F,求与之等值的等额年金A。\\
\textbf{偿债基金公式:}
$$A=F[\frac{i}{(1+i)^n-1}]$$
\textbf{偿债基金系数:}
$$A = F ( A / F , i , n )$$
\textbf{例题:}某人5年后需要10万元,他准备从现在起每年年末向银行存入一笔等额金额,已知存款年利率为5\%,复利计息,问每年的等额存款是多少?\\
\textbf{答案:}$A=100000(A/F,5\%,5)=100000 \times 0.18097)=18097$(元)\\
\textbf{引申:}如果上例改为每年年初向银行存入一笔等额金额,则每年的等额存款是多少?(提示:先付年金终值公式的逆运算)\\
\textbf{答案:}$A=100000(A/F,5\%,5)/(1 + 5\%)= (100000 \times 0.18097)/1.05= 17235$(元)

\subsubsection{等额分付年金现值公式(AP)}
若在每年年末等额分付资金$A$,在利率为$i$的条件下与之经济等值的现值为多少?其现金流量图\ref{fig:9}如图所示。

\begin{figure}[H]
    \centering
    \includegraphics[width=0.8\textwidth]{image/等额收付现值现金流量图.jpg}
    \caption{等额收付现值现金流量图}
    \label{fig:9}
\end{figure}

\textbf{年金现值公式:}
$$P=A[\frac{(1+i)^n-1}{i(1+i)^n}]$$

\textbf{年金现值系数:}
$$P=A(P/A,i,n)$$
\textbf{例题:}某人为了在以后3年内每年年末可以从银行提取1000元,假设存款年利率为10\%,复利计息,问该人现在应该存入多少钱?\\
\textbf{答案:}$P=A(P/A,10\%,3)=1000 \times 2.487=2487$(元)\\
\textbf{引申:}上例中,如果改为“为了在以后3年内每年年初可以从提取1000元”,其他条件不变,则该人现在应该存入多少钱?\\
\textbf{答案:}$P_d=A \cdot (P/A,10\%,3) \cdot (1+i) =1000 \times 2.487 \times (1+10\%)=2735.7$(元)

\textbf{先付年金现值公式}
$$P_d=A[\frac{(1+i)^n-1}{i(1+i)^n}](1+i)$$
$$P_d=A(P/A,i,n)(1+i)$$

\textbf{特例——永续年金现值公式}
$$P=\frac{A}{1+i}+\frac{A}{(1+i)^2}+...\frac{A}{(1+i)^n}+...=A\lim_{n \to \infty}[\frac{(1+i)^n-1}{i(1+i)^n}]=\frac{A}{i}$$

\subsubsection{资本回收公式($P \to A$)}
资本回收公式是年金现值公式的逆运算,即已知现值$P$,求与之等值的年金$A$。

$$A=P[\frac{i(1+i)^n}{(1+i)^n-1}]$$

\textbf{资本回收系数:}
$$A=P(A/P,i,n)$$
\textbf{例题:}某人以分期付款的方式买下一套价值20万元的房子,年利率为8\%,付款期限为15年,每年付款额相等,问此人每年年末需要付多少钱?\\
\textbf{答案:}$A=200000(A/P,8\%,15)=200000 \times 0.11683=23367$(元)\\
\textbf{引申:}在上例中,如问此人每年年初需要付多少钱?(提示:先付年金现值公式的逆运算)\\
\textbf{答案:}$A=200000(A/P,8\%,15)/(1+i)=(200000 \times 0.11683)/(1+8\%)=21636$(元)


\chapter{基准折现率的确定}
基准折现率(基准收益率)是衡量项目可行性和方案比选的主要依据。它反映投资者对项目占用资金时间价值的判断,是投资者从事投资活动可接受的下临界值。其高低直接影响着经济评价决策指标的计算结果。如从$NPV(i)$的特性看,$i$定的越大,$NPV$越小,表明对项目要求越严,审查越严。

一个例子,你去玩一个游戏,赢了这个游戏给你100块钱,输了就什么都没有,输赢的概率都是50\%。此时给你两种选择:要么玩;要么你不参加游戏,直接给你50块钱。你的选择是?

那么,如果你去玩一个游戏,赢了这个游戏给你200块钱,输了就什么都没有,输赢的概率都是50\%。还是同样的两种选择:要么玩;要么你不参加游戏,直接给你50块钱。你的选择又是什么呢?

因此,我们确定基准折现率时,需考虑的因素如下:
机会成本、风险、资本成本、通胀。

\section{风险与收益}
\subsection{风险的概念}
风险是指投资者未来收益的不确定性。
\subsection{风险投资原则}
\begin{itemize}
    \item 同风险、同收益;
    \item 高风险,高收益;
    \item 低风险,低收益。
\end{itemize}

\subsection{股票、债券的报酬率}
\subsubsection{股票的报酬率}
$$\mbox{报酬率}=\mbox{股利收益率}+\mbox{资本利得收益率}=\frac{\mbox{期末支付的股利}+\mbox{当期市场价值的变动}}{\mbox{期初市场价值}}$$

\subsubsection{债券的报酬率}
通过长期的观察发现,债券的报酬率远低于股票的报酬率,国库券的报酬率很低。如买政府的国库券,几乎没有任何违约风险,此时的收益率可看成无风险报酬率。

\subsubsection{风险溢酬}
风险非常大的股票的报酬率和国库券的报酬率差额是超额报酬率。可以被解释为承担风险的报酬,称为\textbf{风险溢酬}。风险性资产会赚取风险溢酬,即承担风险就会有回报。而且通过对股票、债券等计算其\textbf{方差和标准差},发现报酬率越高,其标准差就越大,也就是在某一给定的年度,股票价值出现特别大的变动的机会就很大,说明收益率越高,风险越大。
$$\mbox{风险溢酬}=\mbox{期望报酬率}-\mbox{无风险报酬率}$$

\subsection{投资组合的报酬率及风险}

\begin{figure}[H]
    \centering
    \includegraphics[width=0.9\textwidth]{image/投资组合的报酬率及风险.png}
    \caption{投资组合的报酬率及风险}
    \label{fig:10}
\end{figure}

\subsection{系统风险和非系统风险}
\textbf{系统风险(市场风险)}:对整个市场有影响,比如总体经济状况的不确定性,利率、经济衰退、战争等,属不可分散风险。

\textbf{非系统风险(特有风险或者具体资产风险)}:影响某个单项资产或一小组资产,是个别公司或者资产特有的,属可分散风险。
$$\mbox{整体风险}=\mbox{系统风险}+\mbox{非系统风险}$$

通过分散化化解非系统风险几乎没有任何成本,因此承担这种风险没有回报。
承担风险时所得到的回报的大小,仅取决于系统风险。也即一项资产的期望报酬率取决于系统风险。

通常用$\beta$衡量不同投资的系统风险水平。平均资产的$\beta$为1。

\begin{figure}[H]
    \centering
    \includegraphics[width=0.8\textwidth]{image/CAPM中每月平均收益率和Beta值的检验.jpg}
    \caption{CAPM中每月平均收益率和Beta值的检验}
    \label{fig:11}
\end{figure}
$$\mbox{期望报酬率}=\mbox{无风险报酬率}+\mbox{市场风险溢酬}$$
$$R_E=R_f+[E(R_M)-R_f]*B_i$$

$R_f$:无风险报酬率;

$[E(R_M)-R_f]$:市场平均风险溢酬;

$B_i$:一项特定资产相对于平均资产而言,所面临的系统风险的大小。

\section{资本成本的确定}
\subsection{企业筹资的类别}
\textbf{权益资金(Equity)}是由企业所有者投入的资金,企业可长期使用,无需偿还。

\textbf{债务资金(Debt)}是由企业债权人投入的资金,企业需按约使用,按期偿还。

资本成本(cost of capital)是指投资者投资于企业的必要预期收益率。因此,资本成本与投资者投资于企业所承担的风险有关。

权益投资者承担相对较高的风险,因此权益资本成本较大。权益投资者承担的风险用贝塔系数来衡量。

债务投资者(债权人)承担相对较低的风险,因此债务成本较低。

\subsection{债务资本成本的计算}
\subsubsection{债券定价公式}
$$P=\sum_{t=1}^{n} \frac{I}{(1+r_D)^t}+\frac{F}{(1+r_D)^n}$$

$I$:各期的利息,等于票面值乘以票面利率。

$F$:债券面值。

$n$:债券到期期限。

$r_D$:债券的到期收益率,或债券资本成本。\\
\textbf{例题:}A公司发行的债券,面值为1000元,票面利率10\%,期限为10年。目前该债券还有8年到期,当前的市场价格为1050元。计算该债券的到期收益率或资本成本。\\
\textbf{答案:}
$$1050=\sum_{t=1}^{8}\frac{100}{(1+r_D)^t}+\frac{1000}{(1+r_D)^8}$$
$$r_D=9\%$$









\end{document}
