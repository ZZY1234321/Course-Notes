\chapter{经济效果评价方法}
决策对象只有一个方案时,则可用NPV、IRR等指标评价标准进行该方案的选取。当决策对象存在多个方案需要进行选取时,孰优孰劣?需要依据其决策结构的不同而选用不同方法。多个方案之间的关系:独立方案、互斥方案、相关方案。

\section{独立方案的经济效果评价}
\subsection{概念-独立方案}
作为评价对象的各个方案的现金流是独立的,不具有相关性,且任一方案的采用与否都不影响其他方案是否采用的决策。

\subsection{独立方案的评价}
独立方案的采用与否,只取决于方案自身的经济性,即只需检验各个方案能否通过NPV、NAV、IRR等指标的评价标准。

\textbf{“绝对经济效果检验”}

\subsection{举例:}两个独立方案A和B,其现金流如表所示,试判断其经济性($i_0$=15\%)

\begin{table}[H]
\centering
\begin{tabular}{|c|c|c|}
\hline
t   & 0 & 1-10  \\ \hline
A   & -200 & 45 \\ \hline
B   & -200 & 30 \\ \hline
\end{tabular}
\caption{独立方案A、B的净现金流量(单位:万元)}
\end{table}
\noindent{\textbf{答案:}\\
(1)本例为独立方案,先计算方案自身的绝对效果指标,NPV、或者 NAV、IRR,再进行判断。}\\
$NPV_A=-200+45(P/A,15\%,10)=25.8$万元\\
$NPV_B=-200+30(P/A,15\%,10)=-49.4$万元\\
按NPV判断标准, $NPV_A  \textgreater 0$, $NPV_B \textless 0$,所以A方案予以接受,B方案应予拒绝。\\
(2) $NAV_A= NPV_A(A/P,15\%,10)=-200(A/P,15\%,10)+45=5.14$万元\\
$NAV_B= NPV_B(A/P,15\%,10)=-200(A/P,15\%,10)+30=-9.85$万元\\
按NAV判断准则,$NAVA \textgreater 0$, $NAVB \textless 0$,所以A方案予以接受,B方案应予拒绝。\\
(3)设A方案内部收益率为$IRR_A$,B方案内部收益率为$IRR_B$,由方程:\\
$$-200+45(P/A,IRR_A,10)=0$$
$$-200+30(P/A,IRR_B,10)=0$$
\noindent 解得:\\
$IRR_A=18.3\%,IRR_B=8.1\%,i_0=15\%$。据IRR判别标准,所以A予以接受,B拒绝。

\subsection{结论}
对于独立方案而言,经济上是否可行的判据是其绝对经济效果指标是否优于一定的检验标准。无论采用哪种评价指标,评价结论都是相同的。

\section{互斥方案的经济效果评价}
\subsection{概念-互斥方案}
方案之间存在互不相容、相互排斥关系的称为互斥方案,在对多个互斥方案进行比选时,至多只能选取其中之一。\textbf{“有你无我” 、“你死我活”}。
\subsection{互斥方案经济效果评价的内容}
一是考察各个方案自身的经济效果,即进行“绝对经济效果检验”;二是考察哪个方案相对最优,称“相对经济效果检验”。先看自身条件,再与候选比优!
\subsection{互斥方案经济效果评价特点}
要进行方案之间的比选;参加比选的方案之间应具有可比性。

如:考察时间段及计算期的可比性;方案风险水平的可比性;评价所使用假定的可比性。

\subsection{互斥方案的分类}
按互斥方案的寿命期是否相等,分为寿命期相等的互斥方案和寿命期不等的互斥方案。

\subsection{寿命期相等的互斥方案比选方法-增量分析法}
\subsubsection{引例}
由于理论不好直接理解,这里引入一个例子,从例子理解这个比选方法。\\
\textbf{例子:}互斥方案D、E的净现金流如下表所示,基准收益率为10\%,试判断应选择哪个方案?
\begin{table}[H]
\centering
\begin{tabular}{|c|c|c|}
\hline
年份   & 0 & 1-5  \\ \hline
方案D   & -100 & 35 \\ \hline
方案E   & -150 & 50 \\ \hline
\end{tabular}
\end{table}
\noindent{\textbf{解析:}}\\
(1)首先,进行“绝对经济效果检验”,计算两个方案的绝对经济效果指标NPV和IRR:
$NPV_D=-100+35 \times (P/A,10\%,5)=32.68$(万元)\\
$NPV_E=-150+50 \times (P/A,10\%,5)=39.54$(万元)\\
由方程:\\
$-100+35 \times (P/A,IRR_D,5)=0$\\
$-150+50 \times (P/A,IRR_E,5)=0$\\
解得: $IRR_D=22\%$,$IRR_E=20\%$\\
结论:方案D、E都能通过绝对经济效果检验。\\
(2)其次, 进行“相对经济效果检验”,按一定判别标准,比较哪个方案相对最优;\\
NPV法: $NPV_E \textgreater NPV_D$,若按NPV最大准则,方案E优于方案D;\\
IRR法: $IRR_E \textless IRR_D$。\\
若按IRR最大为比选准则,方案D优于方案E。\\
结论:以上采用不同指标,在一定评选准则下,进行相对评选,方案选择结果相互矛盾。

\textbf{矛盾的解决:}分析投资方案的本质在于增量投资的经济性。投资目的是获取最大收益,互斥方案的比选实际上是判断额外投资是否合理。较大投资的方案可分为两部分:第一部分与小投资方案相同,第二部分是增量投资的收益。如果增量投资的收益较好,则较大投资方案优;否则,小投资方案优。

\begin{table}[H]
\centering
\begin{tabular}{|c|c|c|}
\hline
年份   & 0 & 1-5  \\ \hline
方案D   & -100 & 35 \\ \hline
方案E   & -150 & 50 \\ \hline
差额现金流   & -50 & 15 \\ \hline
\end{tabular}
\end{table}

差额现金流即增量净现金流,通过ΔNCF,可以计算ΔNPV(增量净现值)或ΔIRR(增量内部收益率)。

差额投资收益是否满意的比选标准:ΔNPV>0,ΔIRR>$i_0$(单一方案比选标准)

上例中,ΔNPV=$-50+15(P/A,10\%,5)=NPV_E-NPV_D=39.54-32.68=6.86 \textgreater 0$

由方程-50+15(P/A,ΔIRR,5)=0可解得ΔIRR=15.2\%>10\%

相对比选结论:E优(增量投资有满意的增量收益)

最终结论:E优(同时通过绝对/相对经济效果检验)

\subsubsection{概念-增量分析法}
通过计算ΔNCF(增量净现金流)的经济效果评价指标ΔNPV(增量净现值)、ΔIRR(增量内部收益率)等来评价增量投资经济效果,从而对\textbf{投资额不等的互斥方案}进行比选的方法,称为\textbf{增量分析法(差额分析法)}(互斥方案比选的基本方法)。

在对互斥方案进行比选时,分别用ΔNPV 、 ΔIRR 、NPV(最大)、 NAV(最大)、 PC、AC(最小)、等指标的判别准则进行方案选优时,结论是一致的,正确的;但IRR最大准则,不能总保证比选结论的正确性。当ΔIRR \textless $i_0$,IRR最大准则成立;ΔIRR ≥ $i_0$, IRR最大准则不成立。

\section{寿命期不等的互斥方案比选方法}
绝对经济效果检验:可采用净现值法
如何进行相对比较?假设各方案可以按原有规模不断重复。以互斥方案寿命期最小公倍数作为比较期,最短计算期法,以互斥方案无限循环期作为比较期,以一年作为比较期。

\textbf{年值法-最简便(避免“期”不同问题)}\\
NAV(净年值):最大且非负为最优方案;\\
AC(费用年值):最小为最优方案。

\textbf{年值折现法-设定共同分析(计算)期}\\
首先将各个方案的年值按某一共同的分析期折现,得到用于方案比选的现值(净现值/费用现值)。再根据净现值/费用现值的比选准则进行方案选取。

\section{相关方案比较(了解即可)}
在多方案之间,如果接受(或拒绝)某一方案,会显著改变其他方案的现金流量,或会影响其他方案的接受(或拒绝),就说这些方案是相关的。

\subsection{种类}
\textbf{1、完全互斥型:}\\
接受某一方案,就必须放弃其他方案。“有你无我”(在前面已经有详细的讲解)

\textbf{2、相互依存型和完全互补型}\\
含义:如果两个或多个方案之间,某个方案的实施要求以另一方案(或另几个方案)的实施为条件,则这两个方案(或若干个)方案具有相互依存性,或者说具有完全互补性。

举例:在两个不同的军工厂分别建设生产新型火炮和与之配套的炮弹的项目。

\textbf{3、现金流相关型}\\
含义:如果若干方案中,任一方案的取舍会导致其他方案现金流的变化,则这些方案之间为现金流相关型。

举例:A方案是在某大河上建一座收费公路桥,B方案是在桥址附近建收费轮渡码头,任一方案的实施或放弃都会影响另一方案的收入。

\textbf{4、资金约束导致的方案相关}\\
含义:在资金有限的情况下,接收某些方案则意味着不得不放弃另外一些方案。

\textbf{5、混合相关型}\\
含义:在方案众多的情况下,方案间的相互关系可能包括多种类型,称为混合相关型。

\section{本章重点总结}

\subsection{独立方案决策}
NPV $\geq 0$、IRR $\geq i_0$

\subsection{互斥方案决策(最为重要)}
\subsubsection{1. 收益现金流项目}
\textbf{(1)寿命期相等:重要}
\begin{itemize}
    \item 选择NPV $\geq 0$,且NPV最大的项目;
    \item 如果IRR $ \geq i_0$,则选择投资额较大的项目;如果IRR $< i_0$,则选择投资额较小的项目。
\end{itemize}

\textbf{(2)寿命期不等:}
\begin{itemize}
    \item \textbf{年值法:选择NAV最大且大于零的项目;重要}
    \item 寿命期最小公倍数法:选择NPV(N)最大且大于零的项目;
    \item 无限循环法:选择NPV大于零,且NPV最大的方案。
\end{itemize}

\subsubsection{2. 费用现金流项目}
\textbf{(1)寿命期相等:}

\textbf{选择PC或AC最小的项目。重要}

\textbf{(2)寿命期不等:}
\begin{itemize}
    \item 选择寿命期最小公倍数内PC最小的项目;
    \item 选择AC最小的项目。重要
\end{itemize}

\subsection{现金流相关方案决策(了解)}
\subsubsection{互斥方案组合法}
\subsection{资本限量决策(了解)}
\subsubsection{1. 互斥方案组合法}
\subsubsection{2. “净现值+剩余资金”法}